\title{Introduction to Design}

\paragraph{Introduction to Design: Certified True Randomizers} \hspace{0pt} \\


For this assignment we will use the \href{https://www.random.org/app/}{Randomizers} app as the example.


\begin{enumerate}
  \item \textbf{What gap in the user experience can you infer from the app’s functionality?}

    The user needed a simple way to generate randomness, including but not limited to \hyphenation{fifty fifty} results.

    Of course, this could easily be solved by flipping an actual coin, but this can be messy;
    first you have to get a coin out of your wallet, flip it, hope you'll catch it alright, and then you have put everything back again.
    for this we're looking for a cleaner, more reliable solution that is just as quick.

    There are a whole array of similar apps, webservices already available, but the main selling point of this one is that it is the ``official'' app
    for \url{https://www.random.org}, a site famous for it's ``true'' randomness. We can assume that this app will therefore have that same focus.

    So the primary function of this app is to produce reliably random results in \hyphenation{fifty fifty} chance scenarios, similar to flipping a coin.
    Any additional desiriable qualities would be:

    \begin{itemize}
      \item quicker and cleaner than an actual coin toss
      \item visual representation of the result
      \item coverage for other chance scenarios like dice rolling, name picking etc.
      \item \href{https://www.xkcd.com/221/}{must be truly random}
      \item native mobile app
    \end{itemize}

  \item \textbf{Do you think the designer really understands the problem?}

    This app has the advantage of basically being a mobile native front-end for an already succesful webservice. 
    They therefore didn't have to deal with the problem of creating ``true randomness'' on predictable machines like computers, 
    and so the designer could put all their effort into figuring out how to succesfully relay all the information.

    The \href{https://www.random.org}{webservice} is hardly useable on mobile devices. 

    Not only is the design of the is very oldschool to begin with, and hardly optimized for smaller screens, but the most used services
    are almost hidden in difficult to decern links and menus.
    The coin tosser also defaults to tossing two coins, which is a very rare case scenario.

    In this the app definitely does a much better job in getting the user where they want to go. 
    
    You can start tossing coins right away, and easily switch between chance scenarios by sliding left or right. 
    Accounting for the time to take your phone out and start the app, it is easily just as quick as tossing a real coin, let alone with other scenarios.

    This app however does fetch all it's results from random.org, so that speed comes with the condition that you must be online. 
    This is the price the user has to pay for ``true randomness'', 
    and if that isn't such a big issue for the user, definitely a point where the competition can get ahead.


  \item \textbf{What market segments can you define for this app?}
    
    This app can be used by a wide array of people who often encounter scenario’s that have to be decided by chance. 
    This can be sports refererees, players of table top games, or people who just like to leave things to chance. 

    However there is competition in this field, and with the affiliation to random.org we can assume this app 
    specifically targets people who are in the know about the difficulty to create randomness on computers, and who care about “true randomness”.

 
  \item \textbf{List examples of needs that could be fulfilled for differe nt market segments served by this app.}
    
    Several possibe needs would be:
    \begin{itemize}
      \item fast
      \item native
      \item multiple scenario’s often encountered in (daily) life
      \item truly random
    \end{itemize}

    As mentioned, there is a lot of competition in this field. Most apps though go little further than simple random number generation, 
    maybe with within custom range, so they already fail to translate that to easily recognizeable scenario’s like the coin toss and dice roll.

    Only one other app provides similar features to ours, and more. 
    The similarly named \href{https://play.google.com/store/apps/details?id=com.blogspot.truerandomgenerator}{True Random Generator} not only provides 
    many of the same features as Randomizer but also some other ones, including a “safe password generator”, 
    another often encountered scenario where randomness plays a big role.

    To top it off, this app promises similiar ``true random'' output, but manages to generate this natively, removing the need for a connection to the internet.
    It is hard to say though whether their random is as strong as that of Randomizer.


  \item \textbf{Prepare and submit one discussion question about the chapters above.}
    
    How does Ulrich’s theory of aesthetics relate to non-physical artifacts? How does it compare to the Google Design Guide and Material for example?


\end{enumerate}



% vim:set ts=2 sw=2 et:
