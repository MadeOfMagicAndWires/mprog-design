\documentclass{article}

\usepackage{ifpdf}
\usepackage[utf8]{inputenc}
\usepackage{hyperref}

\title{Introduction to Design}
\author{Joost Bremmer}
\date{\today}

\begin{document}


\paragraph{Certified True Randomizers} \hspace{0pt} \\


For this assignment we will use the \href{https://www.random.org/app/}{Randomizers} app as the example.


\begin{enumerate}
  \item What gap in the user experience can you infer from the app’s functionality?

    The user needed a simple way to generate randomness, including but not limited to \hyphenation{50 50} results.
    Of course, this could easily be solved by flipping an actual coin, but this can be messy;
    first you have to get a coin out of your wallet, flip it, hope you'll catch it alright, and then you have put everything back again.
    for this we're looking for a cleaner, more reliable solution that is just as quick.

    There are a whole array of similar apps, webservices already available, but the main selling point of this one is that it is the ``official'' app
    for \url{https://www.random.org}, a site famous for it's ``true'' randomness. We can assume that this app will therefore have that same focus.

    So the primary function of this app is to produce reliably random results in \hyphenation{50 50} chance scenarios, similar to flipping a coin.
    Any additional desiriable qualities would be:

    \begin{itemize}
      \item quicker and cleaner than an actual coin toss
      \item visual representation of the result
      \item coverage for other chance scenarios like dice rolling, name picking etc.
      \item /href{https://www.xkcd.com/221/}{must be truly random}
      \item native mobile app
    \end{itemize}

  \item Do you think the designer really understands the problem?
    
    This app has the advantage of basically being a mobile native front-end for an already succesful webservice. 
    They therefore didn't have to deal with the problem of creating ``true randomness'' on predictable machines like computers, 
    and so the designer could put all their effort into figuring out how to succesfully relay all the information.

    The \href{https://www.random.org}{webservice} is hardly useable on mobile devices. 

    Not only is the design of the is very oldschool to begin with, and hardly optimized for smaller screens, but the most used services
    are almost hidden in difficult to decern links and menus.
    The coin tosser also defaults to tossing two coins, which is a very rare case scenario.

    In this the app definitely does a much better job in getting the user where they want to go. 
    
    You can start tossing coins right away, and easily switch between chance scenarios by sliding left or right. 
    Accounting for the time to take your phone out and start the app, it is easily just as quick as tossing a real coin, with much less of a hassle.
    It also 


  \item What market segments can you define for this app?
  \item List examples of needs that could be fulfilled for different market segments served by this app.
  \item Prepare and submit one discussion question about the chapters above.
\end{enumerate}

\end{document}



% vim:set ts=2 sw=2 et:
